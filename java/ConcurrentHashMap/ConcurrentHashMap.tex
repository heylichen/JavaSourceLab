\documentclass[UTF8,11pt]{ctexbook}


\usepackage{titlesec}		%标题样式
\usepackage{geometry}		%页面间距
\usepackage{microtype}		%单词间距自动调整
\usepackage{graphicx}		%插入图片   
\usepackage{mathptmx}		%使用times字体
\usepackage{fancyhdr}		%页眉页脚,不用中文
\usepackage{clrscode3e}		%clrs package
\usepackage{lipsum} % for dummy text
\usepackage{enumitem}

 

%缩进
\setlength\parindent{0pt}
%行间距
\linespread{1}
%页眉
\pagestyle{fancy}
\renewcommand{\chaptermark}[1]{ \markboth{#1}{} }	%页面用英文, 小写
\renewcommand{\sectionmark}[1]{ \markright{#1}{} }

\renewcommand\headrulewidth{0pt}
\lhead{chapter \thechapter \ \leftmark}
\rhead{\thesection \ \rightmark}
%页面间距
\newgeometry{left = 3 cm, right=3 cm,top=1.5cm,bottom=1.5cm}
%文中一些字体
\newcommand{\bold}[1]{\textbf{#1}}
\newcommand{\bolditalic}[1]{\textbf{\textit{#1}}}
\newcommand{\point}[1]{\noindent{\textbf{#1}}}

\newenvironment{myitemize}
{ \begin{itemize}
		\setlength{\itemsep}{0pt}
		\setlength{\parskip}{0pt}
		\setlength{\parsep}{0pt}     }
	{ \end{itemize}                  }

\newenvironment{myenumerate}
{ \begin{enumerate}
		\setlength{\itemsep}{0pt}
		\setlength{\parskip}{0pt}
		\setlength{\parsep}{0pt}     }
	{ \end{enumerate}                  }


%重定义chapter, section的样式及行间距,不用cbook的默认样式
\titlespacing{\chapter}{0pt}{2.5ex plus 1ex minus .2ex}{1.3ex plus .2ex}
\titleformat{\chapter}{\huge\bfseries}{\thechapter}{2em}{}
\titleformat{\section}{\Large\bfseries}{\thesection}{1em}{}

\usepackage{listings}
\usepackage{color}
\usepackage{xcolor}
\definecolor{dkgreen}{rgb}{0,0.6,0}
\definecolor{gray}{rgb}{0.5,0.5,0.5}
\definecolor{mauve}{rgb}{0.58,0,0.82}
\lstset{frame=tb,
	language=Java,
	aboveskip=3mm,
	belowskip=3mm,
	showstringspaces=false,
	columns=flexible,
	basicstyle = \ttfamily\small,
	numbers=none,
	numberstyle=\tiny\color{gray},
	keywordstyle=\color{blue},
	commentstyle=\color{dkgreen},
	stringstyle=\color{mauve},
	breaklines=true,
	breakatwhitespace=true,
	tabsize=3
}
 


\begin{document}
	\chapter{Class ConcurrentHashMap$<$K,V$>$}
	\section{javaDoc}
    java.lang.Object\\   
    \ java.util.AbstractMap$<$K,V$>$\\  
    \  java.util.concurrent.ConcurrentHashMap$<$K,V$>$ \\  
	
	All Implemented Interfaces:\\
	Serializable, ConcurrentMap$<$K,V$>$, Map$<$K,V$>$\\
	
	A hash table supporting full concurrency of retrievals and high expected concurrency for updates. This class obeys the same functional specification as Hashtable, and includes versions of methods corresponding to each method of \bold{Hashtable}. However, even though all operations are thread-safe, retrieval operations do not entail locking, and there is not any support for locking the entire table in a way that prevents all access. This class is fully interoperable with Hashtable in programs that rely on its thread safety but not on its synchronization details.
	\\
	
	Retrieval operations (including get) generally do not block, so may overlap with update operations (including put and remove). Retrievals reflect the results of the most recently completed update operations holding upon their onset. (More formally, an update operation for a given key bears a happens-before relation with any (non-null) retrieval for that key reporting the updated value.) For aggregate operations such as putAll and clear, concurrent retrievals may reflect insertion or removal of only some entries. Similarly, Iterators, Spliterators and Enumerations return elements reflecting the state of the hash table at some point at or since the creation of the iterator/enumeration. They do not throw ConcurrentModificationException. However, iterators are designed to be used by only one thread at a time. Bear in mind that the results of aggregate status methods including size, isEmpty, and containsValue are typically useful only when a map is not undergoing concurrent updates in other threads. Otherwise the results of these methods reflect transient states that may be adequate for monitoring or estimation purposes, but not for program control.
	\\
	
	The table is dynamically expanded when there are too many collisions (i.e., keys that have distinct hash codes but fall into the same slot modulo the table size), with the expected average effect of maintaining roughly two bins per mapping (corresponding to a 0.75 load factor threshold for resizing). There may be much variance around this average as mappings are added and removed, but overall, this maintains a commonly accepted time/space tradeoff for hash tables. However, resizing this or any other kind of hash table may be a relatively slow operation. When possible, it is a good idea to provide a size estimate as an optional initialCapacity constructor argument. An additional optional loadFactor constructor argument provides a further means of customizing initial table capacity by specifying the table density to be used in calculating the amount of space to allocate for the given number of elements. Also, for compatibility with previous versions of this class, constructors may optionally specify an expected concurrencyLevel as an additional hint for internal sizing. Note that using many keys with exactly the same hashCode() is a sure way to slow down performance of any hash table. To ameliorate impact, when keys are Comparable, this class may use comparison order among keys to help break ties.\\
	
	A \bold{Set} projection of a \bold{ConcurrentHashMap} may be created (using \bold{newKeySet()} or \bold{newKeySet(int)}), or viewed (using \bold{keySet(Object)} when only keys are of interest, and the mapped values are (perhaps transiently) not used or all take the same mapping value.\\
	
	A ConcurrentHashMap can be used as scalable frequency map (a form of histogram or multiset) by using LongAdder values and initializing via computeIfAbsent. For example, to add a count to a ConcurrentHashMap$<$String,LongAdder$>$ freqs, you can use freqs.computeIfAbsent(k -$>$ new LongAdder()).increment();\\
	
	This class and its views and iterators implement all of the optional methods of the Map and Iterator interfaces.\\
	
	Like Hashtable but unlike HashMap, this class does not allow null to be used as a key or value.\\
	
	ConcurrentHashMaps support a set of sequential and parallel bulk operations that, unlike most Stream methods, are designed to be safely, and often sensibly, applied even with maps that are being concurrently updated by other threads; for example, when computing a snapshot summary of the values in a shared registry. There are three kinds of operation, each with four forms, accepting functions with Keys, Values, Entries, and (Key, Value) arguments and/or return values. Because the elements of a ConcurrentHashMap are not ordered in any particular way, and may be processed in different orders in different parallel executions, the correctness of supplied functions should not depend on any ordering, or on any other objects or values that may transiently change while computation is in progress; and except for forEach actions, should ideally be side-effect-free. Bulk operations on Map.Entry objects do not support method setValue.
	
	\begin{myitemize}
		\item forEach: Perform a given action on each element. A variant form applies a given transformation on each element before performing the action.
		
		\item search: Return the first available non-null result of applying a given function on each element; skipping further search when a result is found.
		
		\item reduce: Accumulate each element. The supplied reduction function cannot rely on ordering (more formally, it should be both associative and commutative). There are five variants:
				\begin{myitemize}
					\item Plain reductions. (There is not a form of this method for (key, value) function arguments since there is no corresponding return type.)
					\item Mapped reductions that accumulate the results of a given function applied to each element.
					\item Reductions to scalar doubles, longs, and ints, using a given basis value.
				\end{myitemize}
	\end{myitemize}  
	
	These bulk operations accept a parallelismThreshold argument. Methods proceed sequentially if the current map size is estimated to be less than the given threshold. Using a value of $Long.MAX\_VALUE$ suppresses all parallelism. Using a value of 1 results in maximal parallelism by partitioning into enough subtasks to fully utilize the ForkJoinPool.commonPool() that is used for all parallel computations. Normally, you would initially choose one of these extreme values, and then measure performance of using in-between values that trade off overhead versus throughput.\\
	
	The concurrency properties of bulk operations follow from those of ConcurrentHashMap: Any non-null result returned from get(key) and related access methods bears a happens-before relation with the associated insertion or update. The result of any bulk operation reflects the composition of these per-element relations (but is not necessarily atomic with respect to the map as a whole unless it is somehow known to be quiescent). Conversely, because keys and values in the map are never null, null serves as a reliable atomic indicator of the current lack of any result. To maintain this property, null serves as an implicit basis for all non-scalar reduction operations. For the double, long, and int versions, the basis should be one that, when combined with any other value, returns that other value (more formally, it should be the identity element for the reduction). Most common reductions have these properties; for example, computing a sum with basis 0 or a minimum with basis $MAX\_VALUE$.\\
	
	Search and transformation functions provided as arguments should similarly return null to indicate the lack of any result (in which case it is not used). In the case of mapped reductions, this also enables transformations to serve as filters, returning null (or, in the case of primitive specializations, the identity basis) if the element should not be combined. You can create compound transformations and filterings by composing them yourself under this "null means there is nothing there now" rule before using them in search or reduce operations.\\
	
	Methods accepting and/or returning Entry arguments maintain key-value associations. They may be useful for example when finding the key for the greatest value. Note that "plain" Entry arguments can be supplied using new AbstractMap.SimpleEntry(k,v).\\
	
	Bulk operations may complete abruptly, throwing an exception encountered in the application of a supplied function. Bear in mind when handling such exceptions that other concurrently executing functions could also have thrown exceptions, or would have done so if the first exception had not occurred.\\
	
	Speedups for parallel compared to sequential forms are common but not guaranteed. Parallel operations involving brief functions on small maps may execute more slowly than sequential forms if the underlying work to parallelize the computation is more expensive than the computation itself. Similarly, parallelization may not lead to much actual parallelism if all processors are busy performing unrelated tasks.\\
	
	All arguments to all task methods must be non-null.\\
	
	notes:\\
	不允许null key, null value.\\
	null有特殊意义,表示不存在。
	 
	

	\section{Implementation notes.} 
	Overview:\\
	
	The primary design goal of this hash table is to maintain
	concurrent readability (typically method get(), but also
	iterators and related methods) while minimizing update
	contention. Secondary goals are to keep space consumption about
	the same or better than java.util.HashMap, and to support high
	initial insertion rates on an empty table by many threads.\\
	
	This map usually acts as a binned (bucketed) hash table.  Each
	key-value mapping is held in a \bold{Node}.  Most nodes are instances
	of the basic \bold{Node} class with hash, key, value, and next
	fields. However, various subclasses exist: \bold{TreeNodes} are
	arranged in balanced trees, not lists.  \bold{TreeBins} hold the roots
	of sets of \bold{TreeNodes}. \bold{ForwardingNodes} are placed at the heads
	of bins during resizing. \bold{ReservationNodes} are used as
	placeholders while establishing values in computeIfAbsent and
	related methods.  The types \bold{TreeBin}, \bold{ForwardingNode}, and
	\bold{ReservationNode} do not hold normal user keys, values, or
	hashes, and are readily distinguishable during search etc
	because they have negative hash fields and null key and value
	fields. (These special nodes are either uncommon or transient,
	so the impact of carrying around some unused fields is
	insignificant.)\\
	
	The table is lazily initialized to a power-of-two size upon the
	first insertion.  Each bin in the table normally contains a
	list of Nodes (most often, the list has only zero or one Node).
	Table accesses require volatile/atomic reads, writes, and
	CASes.  Because there is no other way to arrange this without
	adding further indirections, we use intrinsics
	(sun.misc.Unsafe) operations.\\
	
	We use the top (sign) bit of Node hash fields for control
	purposes -- it is available anyway because of addressing
	constraints.  Nodes with negative hash fields are specially
	handled or ignored in map methods.\\
	
	Insertion (via put or its variants) of the first node in an
	empty bin is performed by just CASing it to the bin.  This is
	by far the most common case for put operations under most
	key/hash distributions.  Other update operations (insert,
	delete, and replace) require locks.  We do not want to waste
	the space required to associate a distinct lock object with
	each bin, so instead use the first node of a bin list itself as
	a lock. Locking support for these locks relies on builtin
	"synchronized" monitors.\\
	
	Using the first node of a list as a lock does not by itself
	suffice though: When a node is locked, any update must first
	validate that it is still the first node after locking it, and
	retry if not. Because new nodes are always appended to lists,
	once a node is first in a bin, it remains first until deleted
	or the bin becomes invalidated (upon resizing).\\
	
	The main disadvantage of per-bin locks is that other update
	operations on other nodes in a bin list protected by the same
	lock can stall, for example when user equals() or mapping
	functions take a long time.  However, statistically, under
	random hash codes, this is not a common problem.  Ideally, the
	frequency of nodes in bins follows a Poisson distribution
	($http://en.wikipedia.org/wiki/Poisson_distribution$) with a
	parameter of about 0.5 on average, given the resizing threshold
	of 0.75, although with a large variance because of resizing
	granularity. Ignoring variance, the expected occurrences of
	list size k are (exp(-0.5) * pow(0.5, k) / factorial(k)). The
	first values are:\\
	
	0:    0.60653066\\
	1:    0.30326533\\
	2:    0.07581633\\
	3:    0.01263606\\
	4:    0.00157952\\
	5:    0.00015795\\
	6:    0.00001316\\
	7:    0.00000094\\
	8:    0.00000006\\
	more: less than 1 in ten million\\
	
	Lock contention probability for two threads accessing distinct
	elements is roughly $1 / (8\ *\ \#elements)$ under random hashes.\\
	
	Actual hash code distributions encountered in practice
	sometimes deviate significantly from uniform randomness.  This
	includes the case when $N > (1<<30)$, so some keys MUST collide.
	Similarly for dumb or hostile usages in which multiple keys are
	designed to have identical hash codes or ones that differs only
	in masked-out high bits. So we use a secondary strategy that
	applies when the number of nodes in a bin exceeds a
	threshold. These TreeBins use a balanced tree to hold nodes (a
	specialized form of red-black trees), bounding search time to
	O(log N).  Each search step in a TreeBin is at least twice as
	slow as in a regular list, but given that N cannot exceed
	($1<<64$) (before running out of addresses) this bounds search
	steps, lock hold times, etc, to reasonable constants (roughly
	100 nodes inspected per operation worst case) so long as keys
	are Comparable (which is very common -- String, Long, etc).
	TreeBin nodes (TreeNodes) also maintain the same "next"
	traversal pointers as regular nodes, so can be traversed in
	iterators in the same way.\\
	
	The table is resized when occupancy exceeds a percentage
	threshold (nominally, 0.75, but see below).  Any thread
	noticing an overfull bin may assist in resizing after the
	initiating thread allocates and sets up the replacement array.
	However, rather than stalling, these other threads may proceed
	with insertions etc.  The use of TreeBins shields us from the
	worst case effects of overfilling while resizes are in
	progress.  Resizing proceeds by transferring bins, one by one,
	from the table to the next table. However, threads claim small
	blocks of indices to transfer (via field transferIndex) before
	doing so, reducing contention.  A generation stamp in field
	sizeCtl ensures that resizings do not overlap. Because we are
	using power-of-two expansion, the elements from each bin must
	either stay at same index, or move with a power of two
	offset. We eliminate unnecessary node creation by catching
	cases where old nodes can be reused because their next fields
	won't change.  On average, only about one-sixth of them need
	cloning when a table doubles. The nodes they replace will be
	garbage collectable as soon as they are no longer referenced by
	any reader thread that may be in the midst of concurrently
	traversing table.  Upon transfer, the old table bin contains
	only a special forwarding node (with hash field "MOVED") that
	contains the next table as its key. On encountering a
	forwarding node, access and update operations restart, using
	the new table.\\
	
	Each bin transfer requires its bin lock, which can stall
	waiting for locks while resizing. However, because other
	threads can join in and help resize rather than contend for
	locks, average aggregate waits become shorter as resizing
	progresses.  The transfer operation must also ensure that all
	accessible bins in both the old and new table are usable by any
	traversal.  This is arranged in part by proceeding from the
	last bin (table.length - 1) up towards the first.  Upon seeing
	a forwarding node, traversals (see class Traverser) arrange to
	move to the new table without revisiting nodes.  To ensure that
	no intervening nodes are skipped even when moved out of order,
	a stack (see class TableStack) is created on first encounter of
	a forwarding node during a traversal, to maintain its place if
	later processing the current table. The need for these
	save/restore mechanics is relatively rare, but when one
	forwarding node is encountered, typically many more will be.
	So Traversers use a simple caching scheme to avoid creating so
	many new TableStack nodes. (Thanks to Peter Levart for
	suggesting use of a stack here.)\\
	
	The traversal scheme also applies to partial traversals of
	ranges of bins (via an alternate Traverser constructor)
	to support partitioned aggregate operations.  Also, read-only
	operations give up if ever forwarded to a null table, which
	provides support for shutdown-style clearing, which is also not
	currently implemented.\\
	
	Lazy table initialization minimizes footprint until first use,
	and also avoids resizings when the first operation is from a
	putAll, constructor with map argument, or deserialization.
	These cases attempt to override the initial capacity settings,
	but harmlessly fail to take effect in cases of races.\\
	
	The element count is maintained using a specialization of
	LongAdder. We need to incorporate a specialization rather than
	just use a LongAdder in order to access implicit
	contention-sensing that leads to creation of multiple
	CounterCells.  The counter mechanics avoid contention on
	updates but can encounter cache thrashing if read too
	frequently during concurrent access. To avoid reading so often,
	resizing under contention is attempted only upon adding to a
	bin already holding two or more nodes. Under uniform hash
	distributions, the probability of this occurring at threshold
	is around $13\%$, meaning that only about 1 in 8 puts check
	threshold (and after resizing, many fewer do so).\\
	
	TreeBins use a special form of comparison for search and
	related operations (which is the main reason we cannot use
	existing collections such as TreeMaps). TreeBins contain
	Comparable elements, but may contain others, as well as
	elements that are Comparable but not necessarily Comparable for
	the same T, so we cannot invoke compareTo among them. To handle
	this, the tree is ordered primarily by hash value, then by
	Comparable.compareTo order if applicable.  On lookup at a node,
	if elements are not comparable or compare as 0 then both left
	and right children may need to be searched in the case of tied
	hash values. (This corresponds to the full list search that
	would be necessary if all elements were non-Comparable and had
	tied hashes.) On insertion, to keep a total ordering (or as
	close as is required here) across rebalancings, we compare
	classes and identityHashCodes as tie-breakers. The red-black
	balancing code is updated from pre-jdk-collections
	($http://gee.cs.oswego.edu/dl/classes/collections/RBCell.java$)
	based in turn on Cormen, Leiserson, and Rivest "Introduction to
	Algorithms" (CLR).\\
	
	TreeBins also require an additional locking mechanism.  While
	list traversal is always possible by readers even during
	updates, tree traversal is not, mainly because of tree-rotations
	that may change the root node and/or its linkages.  TreeBins
	include a simple read-write lock mechanism parasitic on the
	main bin-synchronization strategy: Structural adjustments
	associated with an insertion or removal are already bin-locked
	(and so cannot conflict with other writers) but must wait for
	ongoing readers to finish. Since there can be only one such
	waiter, we use a simple scheme using a single "waiter" field to
	block writers.  However, readers need never block.  If the root
	lock is held, they proceed along the slow traversal path (via
	next-pointers) until the lock becomes available or the list is
	exhausted, whichever comes first. These cases are not fast, but
	maximize aggregate expected throughput.\\
	
	Maintaining API and serialization compatibility with previous
	versions of this class introduces several oddities. Mainly: We
	leave untouched but unused constructor arguments refering to
	concurrencyLevel. We accept a loadFactor constructor argument,
	but apply it only to initial table capacity (which is the only
	time that we can guarantee to honor it.) We also declare an
	unused "Segment" class that is instantiated in minimal form
	only when serializing.\\
	
	Also, solely for compatibility with previous versions of this
	class, it extends AbstractMap, even though all of its methods
	are overridden, so it is just useless baggage.\\
	
	This file is organized to make things a little easier to follow
	while reading than they might otherwise: First the main static
	declarations and utilities, then fields, then main public
	methods (with a few factorings of multiple public methods into
	internal ones), then sizing methods, trees, traversers, and
	bulk operations.\\
	
	
\end{document}